\documentclass[12pt]{article}
\usepackage[top=1in,bottom=1in,left=1in,right=1in]{geometry}
\usepackage{alltt}
\usepackage{array}	
\usepackage{graphicx}
\usepackage{tabularx}
\usepackage{verbatim}
\usepackage{setspace}
\usepackage{listings}
\usepackage{amssymb,amsmath, amsthm}
\usepackage{hyperref}
\usepackage{oz}
\usepackage[cc]{titlepic}
\usepackage{fancyvrb}

\title{SOEN 342 - Sections H and II:\\Software Requirements and Specifications\\
	\ \\
	Project}
\author{Names(s)}
\date{\today}

\begin{spacing}{1.5}
	\begin{document}
		\maketitle
		
		\newpage

		\section{Partial formal specification in Z}
		
		The formal specification of the system introduces the following three types:
		
		\[ SENSOR\_TYPE, LOCATION\_TYPE, TEMPERATURE\_TYPE  \]
		
		\noindent The system's (partial) formal specification is given in the Z language and it consists of schemas and the definitions of operations that constitute the system's exposed interface.
		
	
		\subsection{Schemas}
		
		
		\begin{schema}{TempMonitor}
			deployed~:~\mathbb{P}~SENSOR\_TYPE\\
			map : SENSOR\_TYPE \nrightarrow LOCATION\_TYPE\\
			read : SENSOR\_TYPE  \nrightarrow TEMPERATURE\_TYPE\\
   			locationTemp : LOCATION\_TYPE  \nrightarrow TEMPERATURE\_TYPE\\
			\where
			deployed = \dom map\\
			deployed = \dom read
		\end{schema}
		
		
		
		\begin{schema}{DeploySensorOK}
			\Delta TempMonitor\\
			sensor? : SENSOR\_TYPE\\
			location? : LOCATION\_TYPE\\
			temperature? : TEMPERATURE\_TYPE
			\where
			sensor? \notin deployed\\
			location? \notin \ran map\\
			deployed' = deployed \cup \{ sensor? \}\\
			map' = map \cup \{ sensor? \mapsto location? \}\\
			read' = read \cup \{ sensor? \mapsto temperature? \}\\
   			locationTemp' = locationTemp \cup \{ location? \mapsto temperature? \}
		\end{schema}
		
		
		\begin{schema}{ReadTemperatureOK}
			\Xi TempMonitor\\
			location? : LOCATION\_TYPE\\
			temperature! : TEMPERATURE\_TYPE
			\where
			location? \in \ran map\\
			temperature! = read(map^{-1}(location?))\\
		\end{schema}
		
		
	
		\begin{schema}{Success}
			\Xi TempMonitor\\
			response! : MESSAGE
			\where
			response!~=~'ok'\\
		\end{schema}
		
		
		
		\begin{schema}{SensorAlreadyDeployed}
			\Xi TempMonitor\\
			sensor? : SENSOR\_TYPE\\
			response! : MESSAGE
			\where
			sensor? \in deployed\\
			response!~=~'Sensor~deployed'\\
		\end{schema}
		
		
		
		\begin{schema}{LocationAlreadyCovered}
			\Xi TempMonitor\\
			location? : LOCATION\_TYPE\\
			response! : MESSAGE
			\where
			location? \in \ran map\\
			response!~=~'Location~already~covered'
		\end{schema}
		
		
		
		\begin{schema}{LocationUnknown}
			\Xi TempMonitor\\
			location? : LOCATION\_TYPE\\
			response! : MESSAGE
			\where
			location? \notin \ran map\\
			response!~=~'Location~not~covered'
		\end{schema}
		%% For new operation
		\begin{schema}{LocationTempCollection}
			\Xi TempMonitor\\
			pairs! : LOCATION\_TYPE  \nrightarrow TEMPERATURE\_TYPE
			\where
			pairs! : locationTemp
		\end{schema}
		
		\subsection{Operations}
		
		\[ DeploySensor~\hat{=}~\\
		~~~(DeploySensorOK \wedge Success)~ \oplus\\
		~~~(SensorAlreadyDeployed \vee LocationAlreadyCovered) \]
		
		
		
		\[ ReadTemperature~\hat{=}~\\
		~~~(ReadTemperatureOK \wedge Success) \oplus LocationUnknown \]
		
\end{spacing}
\end{document}